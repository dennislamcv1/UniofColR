% Options for packages loaded elsewhere
\PassOptionsToPackage{unicode}{hyperref}
\PassOptionsToPackage{hyphens}{url}
%
\documentclass[
]{article}
\usepackage{amsmath,amssymb}
\usepackage{iftex}
\ifPDFTeX
  \usepackage[T1]{fontenc}
  \usepackage[utf8]{inputenc}
  \usepackage{textcomp} % provide euro and other symbols
\else % if luatex or xetex
  \usepackage{unicode-math} % this also loads fontspec
  \defaultfontfeatures{Scale=MatchLowercase}
  \defaultfontfeatures[\rmfamily]{Ligatures=TeX,Scale=1}
\fi
\usepackage{lmodern}
\ifPDFTeX\else
  % xetex/luatex font selection
\fi
% Use upquote if available, for straight quotes in verbatim environments
\IfFileExists{upquote.sty}{\usepackage{upquote}}{}
\IfFileExists{microtype.sty}{% use microtype if available
  \usepackage[]{microtype}
  \UseMicrotypeSet[protrusion]{basicmath} % disable protrusion for tt fonts
}{}
\makeatletter
\@ifundefined{KOMAClassName}{% if non-KOMA class
  \IfFileExists{parskip.sty}{%
    \usepackage{parskip}
  }{% else
    \setlength{\parindent}{0pt}
    \setlength{\parskip}{6pt plus 2pt minus 1pt}}
}{% if KOMA class
  \KOMAoptions{parskip=half}}
\makeatother
\usepackage{xcolor}
\usepackage[margin=1in]{geometry}
\usepackage{color}
\usepackage{fancyvrb}
\newcommand{\VerbBar}{|}
\newcommand{\VERB}{\Verb[commandchars=\\\{\}]}
\DefineVerbatimEnvironment{Highlighting}{Verbatim}{commandchars=\\\{\}}
% Add ',fontsize=\small' for more characters per line
\usepackage{framed}
\definecolor{shadecolor}{RGB}{248,248,248}
\newenvironment{Shaded}{\begin{snugshade}}{\end{snugshade}}
\newcommand{\AlertTok}[1]{\textcolor[rgb]{0.94,0.16,0.16}{#1}}
\newcommand{\AnnotationTok}[1]{\textcolor[rgb]{0.56,0.35,0.01}{\textbf{\textit{#1}}}}
\newcommand{\AttributeTok}[1]{\textcolor[rgb]{0.13,0.29,0.53}{#1}}
\newcommand{\BaseNTok}[1]{\textcolor[rgb]{0.00,0.00,0.81}{#1}}
\newcommand{\BuiltInTok}[1]{#1}
\newcommand{\CharTok}[1]{\textcolor[rgb]{0.31,0.60,0.02}{#1}}
\newcommand{\CommentTok}[1]{\textcolor[rgb]{0.56,0.35,0.01}{\textit{#1}}}
\newcommand{\CommentVarTok}[1]{\textcolor[rgb]{0.56,0.35,0.01}{\textbf{\textit{#1}}}}
\newcommand{\ConstantTok}[1]{\textcolor[rgb]{0.56,0.35,0.01}{#1}}
\newcommand{\ControlFlowTok}[1]{\textcolor[rgb]{0.13,0.29,0.53}{\textbf{#1}}}
\newcommand{\DataTypeTok}[1]{\textcolor[rgb]{0.13,0.29,0.53}{#1}}
\newcommand{\DecValTok}[1]{\textcolor[rgb]{0.00,0.00,0.81}{#1}}
\newcommand{\DocumentationTok}[1]{\textcolor[rgb]{0.56,0.35,0.01}{\textbf{\textit{#1}}}}
\newcommand{\ErrorTok}[1]{\textcolor[rgb]{0.64,0.00,0.00}{\textbf{#1}}}
\newcommand{\ExtensionTok}[1]{#1}
\newcommand{\FloatTok}[1]{\textcolor[rgb]{0.00,0.00,0.81}{#1}}
\newcommand{\FunctionTok}[1]{\textcolor[rgb]{0.13,0.29,0.53}{\textbf{#1}}}
\newcommand{\ImportTok}[1]{#1}
\newcommand{\InformationTok}[1]{\textcolor[rgb]{0.56,0.35,0.01}{\textbf{\textit{#1}}}}
\newcommand{\KeywordTok}[1]{\textcolor[rgb]{0.13,0.29,0.53}{\textbf{#1}}}
\newcommand{\NormalTok}[1]{#1}
\newcommand{\OperatorTok}[1]{\textcolor[rgb]{0.81,0.36,0.00}{\textbf{#1}}}
\newcommand{\OtherTok}[1]{\textcolor[rgb]{0.56,0.35,0.01}{#1}}
\newcommand{\PreprocessorTok}[1]{\textcolor[rgb]{0.56,0.35,0.01}{\textit{#1}}}
\newcommand{\RegionMarkerTok}[1]{#1}
\newcommand{\SpecialCharTok}[1]{\textcolor[rgb]{0.81,0.36,0.00}{\textbf{#1}}}
\newcommand{\SpecialStringTok}[1]{\textcolor[rgb]{0.31,0.60,0.02}{#1}}
\newcommand{\StringTok}[1]{\textcolor[rgb]{0.31,0.60,0.02}{#1}}
\newcommand{\VariableTok}[1]{\textcolor[rgb]{0.00,0.00,0.00}{#1}}
\newcommand{\VerbatimStringTok}[1]{\textcolor[rgb]{0.31,0.60,0.02}{#1}}
\newcommand{\WarningTok}[1]{\textcolor[rgb]{0.56,0.35,0.01}{\textbf{\textit{#1}}}}
\usepackage{graphicx}
\makeatletter
\def\maxwidth{\ifdim\Gin@nat@width>\linewidth\linewidth\else\Gin@nat@width\fi}
\def\maxheight{\ifdim\Gin@nat@height>\textheight\textheight\else\Gin@nat@height\fi}
\makeatother
% Scale images if necessary, so that they will not overflow the page
% margins by default, and it is still possible to overwrite the defaults
% using explicit options in \includegraphics[width, height, ...]{}
\setkeys{Gin}{width=\maxwidth,height=\maxheight,keepaspectratio}
% Set default figure placement to htbp
\makeatletter
\def\fps@figure{htbp}
\makeatother
\setlength{\emergencystretch}{3em} % prevent overfull lines
\providecommand{\tightlist}{%
  \setlength{\itemsep}{0pt}\setlength{\parskip}{0pt}}
\setcounter{secnumdepth}{-\maxdimen} % remove section numbering
\ifLuaTeX
  \usepackage{selnolig}  % disable illegal ligatures
\fi
\IfFileExists{bookmark.sty}{\usepackage{bookmark}}{\usepackage{hyperref}}
\IfFileExists{xurl.sty}{\usepackage{xurl}}{} % add URL line breaks if available
\urlstyle{same}
\hypersetup{
  pdftitle={Data Analysis Lab},
  hidelinks,
  pdfcreator={LaTeX via pandoc}}

\title{Data Analysis Lab}
\author{}
\date{\vspace{-2.5em}}

\begin{document}
\maketitle

\hypertarget{assignment-instructions}{%
\subparagraph{Assignment Instructions}\label{assignment-instructions}}

Complete all questions below. After completing the assignment, knit your
document, and download both your .Rmd and knitted output. Upload your
files for peer review.

For each response, include comments detailing your response and what
each line does.

\begin{center}\rule{0.5\linewidth}{0.5pt}\end{center}

\begin{Shaded}
\begin{Highlighting}[]
\FunctionTok{library}\NormalTok{(nycflights13)}
\FunctionTok{library}\NormalTok{(dplyr)}
\end{Highlighting}
\end{Shaded}

\begin{verbatim}
## 
## Attaching package: 'dplyr'
\end{verbatim}

\begin{verbatim}
## The following objects are masked from 'package:stats':
## 
##     filter, lag
\end{verbatim}

\begin{verbatim}
## The following objects are masked from 'package:base':
## 
##     intersect, setdiff, setequal, union
\end{verbatim}

\hypertarget{question-1.}{%
\subparagraph{Question 1.}\label{question-1.}}

Using the nycflights13 dataset, find all flights that departed in July,
August, or September using the helper function between().

\begin{Shaded}
\begin{Highlighting}[]
\NormalTok{flights\_jul\_aug\_sep }\OtherTok{\textless{}{-}}\NormalTok{ flights }\SpecialCharTok{\%\textgreater{}\%}
  \FunctionTok{filter}\NormalTok{(}\FunctionTok{between}\NormalTok{(month, }\DecValTok{7}\NormalTok{, }\DecValTok{9}\NormalTok{))}

\NormalTok{flights\_jul\_aug\_sep}
\end{Highlighting}
\end{Shaded}

\begin{verbatim}
## # A tibble: 86,326 x 19
##     year month   day dep_time sched_dep_time dep_delay arr_time sched_arr_time
##    <int> <int> <int>    <int>          <int>     <dbl>    <int>          <int>
##  1  2013     7     1        1           2029       212      236           2359
##  2  2013     7     1        2           2359         3      344            344
##  3  2013     7     1       29           2245       104      151              1
##  4  2013     7     1       43           2130       193      322             14
##  5  2013     7     1       44           2150       174      300            100
##  6  2013     7     1       46           2051       235      304           2358
##  7  2013     7     1       48           2001       287      308           2305
##  8  2013     7     1       58           2155       183      335             43
##  9  2013     7     1      100           2146       194      327             30
## 10  2013     7     1      100           2245       135      337            135
## # i 86,316 more rows
## # i 11 more variables: arr_delay <dbl>, carrier <chr>, flight <int>,
## #   tailnum <chr>, origin <chr>, dest <chr>, air_time <dbl>, distance <dbl>,
## #   hour <dbl>, minute <dbl>, time_hour <dttm>
\end{verbatim}

\hypertarget{question-2.}{%
\paragraph{Question 2.}\label{question-2.}}

Using the nycflights13 dataset sort flights to find the 10 flights that
flew the furthest. Put them in order of fastest to slowest.

\begin{Shaded}
\begin{Highlighting}[]
\NormalTok{longest\_flights }\OtherTok{\textless{}{-}}\NormalTok{ flights }\SpecialCharTok{\%\textgreater{}\%}
  \FunctionTok{arrange}\NormalTok{(}\FunctionTok{desc}\NormalTok{(distance)) }\SpecialCharTok{\%\textgreater{}\%}
  \FunctionTok{head}\NormalTok{(}\DecValTok{10}\NormalTok{) }\SpecialCharTok{\%\textgreater{}\%}
  \FunctionTok{mutate}\NormalTok{(}\AttributeTok{speed =}\NormalTok{ distance }\SpecialCharTok{/}\NormalTok{ (air\_time }\SpecialCharTok{/} \DecValTok{60}\NormalTok{)) }\SpecialCharTok{\%\textgreater{}\%}  \CommentTok{\# calculate speed in miles per hour}
  \FunctionTok{arrange}\NormalTok{(}\FunctionTok{desc}\NormalTok{(speed))}

\NormalTok{longest\_flights}
\end{Highlighting}
\end{Shaded}

\begin{verbatim}
## # A tibble: 10 x 20
##     year month   day dep_time sched_dep_time dep_delay arr_time sched_arr_time
##    <int> <int> <int>    <int>          <int>     <dbl>    <int>          <int>
##  1  2013     1     6     1019            900        79     1558           1530
##  2  2013     1     7     1042            900       102     1620           1530
##  3  2013     1     3      914            900        14     1504           1530
##  4  2013     1    10      859            900        -1     1449           1530
##  5  2013     1     5      858            900        -2     1519           1530
##  6  2013     1     2      909            900         9     1525           1530
##  7  2013     1     4      900            900         0     1516           1530
##  8  2013     1     9      641            900      1301     1242           1530
##  9  2013     1     8      901            900         1     1504           1530
## 10  2013     1     1      857            900        -3     1516           1530
## # i 12 more variables: arr_delay <dbl>, carrier <chr>, flight <int>,
## #   tailnum <chr>, origin <chr>, dest <chr>, air_time <dbl>, distance <dbl>,
## #   hour <dbl>, minute <dbl>, time_hour <dttm>, speed <dbl>
\end{verbatim}

\hypertarget{question-3.}{%
\paragraph{Question 3.}\label{question-3.}}

Using the nycflights13 dataset, calculate a new variable called
``hr\_delay'' and arrange the flights dataset in order of the arrival
delays in hours (longest delays at the top). Put the new variable you
created just before the departure time.Hint: use the experimental
argument .before.

\begin{Shaded}
\begin{Highlighting}[]
\NormalTok{flights\_with\_hr\_delay }\OtherTok{\textless{}{-}}\NormalTok{ flights }\SpecialCharTok{\%\textgreater{}\%}
  \FunctionTok{mutate}\NormalTok{(}\AttributeTok{hr\_delay =}\NormalTok{ arr\_delay }\SpecialCharTok{/} \DecValTok{60}\NormalTok{) }\SpecialCharTok{\%\textgreater{}\%}
  \FunctionTok{arrange}\NormalTok{(}\FunctionTok{desc}\NormalTok{(hr\_delay)) }\SpecialCharTok{\%\textgreater{}\%}
  \FunctionTok{select}\NormalTok{(year}\SpecialCharTok{:}\NormalTok{day, hr\_delay, dep\_time, }\FunctionTok{everything}\NormalTok{())}

\NormalTok{flights\_with\_hr\_delay}
\end{Highlighting}
\end{Shaded}

\begin{verbatim}
## # A tibble: 336,776 x 20
##     year month   day hr_delay dep_time sched_dep_time dep_delay arr_time
##    <int> <int> <int>    <dbl>    <int>          <int>     <dbl>    <int>
##  1  2013     1     9     21.2      641            900      1301     1242
##  2  2013     6    15     18.8     1432           1935      1137     1607
##  3  2013     1    10     18.5     1121           1635      1126     1239
##  4  2013     9    20     16.8     1139           1845      1014     1457
##  5  2013     7    22     16.5      845           1600      1005     1044
##  6  2013     4    10     15.5     1100           1900       960     1342
##  7  2013     3    17     15.2     2321            810       911      135
##  8  2013     7    22     14.9     2257            759       898      121
##  9  2013    12     5     14.6      756           1700       896     1058
## 10  2013     5     3     14.6     1133           2055       878     1250
## # i 336,766 more rows
## # i 12 more variables: sched_arr_time <int>, arr_delay <dbl>, carrier <chr>,
## #   flight <int>, tailnum <chr>, origin <chr>, dest <chr>, air_time <dbl>,
## #   distance <dbl>, hour <dbl>, minute <dbl>, time_hour <dttm>
\end{verbatim}

\hypertarget{question-4.}{%
\subparagraph{Question 4.}\label{question-4.}}

Using the nycflights13 dataset, find the most popular destinations
(those with more than 2000 flights) and show the destination, the date
info, the carrier. Then show just the number of flights for each popular
destination.

\begin{Shaded}
\begin{Highlighting}[]
\NormalTok{popular\_destinations }\OtherTok{\textless{}{-}}\NormalTok{ flights }\SpecialCharTok{\%\textgreater{}\%}
  \FunctionTok{group\_by}\NormalTok{(dest) }\SpecialCharTok{\%\textgreater{}\%}
  \FunctionTok{filter}\NormalTok{(}\FunctionTok{n}\NormalTok{() }\SpecialCharTok{\textgreater{}} \DecValTok{2000}\NormalTok{) }\SpecialCharTok{\%\textgreater{}\%}
  \FunctionTok{select}\NormalTok{(dest, year, month, day, carrier)}

\NormalTok{number\_of\_flights }\OtherTok{\textless{}{-}}\NormalTok{ popular\_destinations }\SpecialCharTok{\%\textgreater{}\%}
  \FunctionTok{group\_by}\NormalTok{(dest) }\SpecialCharTok{\%\textgreater{}\%}
  \FunctionTok{summarise}\NormalTok{(}\AttributeTok{num\_flights =} \FunctionTok{n}\NormalTok{())}

\FunctionTok{list}\NormalTok{(popular\_destinations, number\_of\_flights)}
\end{Highlighting}
\end{Shaded}

\begin{verbatim}
## [[1]]
## # A tibble: 302,969 x 5
## # Groups:   dest [46]
##    dest   year month   day carrier
##    <chr> <int> <int> <int> <chr>  
##  1 IAH    2013     1     1 UA     
##  2 IAH    2013     1     1 UA     
##  3 MIA    2013     1     1 AA     
##  4 ATL    2013     1     1 DL     
##  5 ORD    2013     1     1 UA     
##  6 FLL    2013     1     1 B6     
##  7 IAD    2013     1     1 EV     
##  8 MCO    2013     1     1 B6     
##  9 ORD    2013     1     1 AA     
## 10 PBI    2013     1     1 B6     
## # i 302,959 more rows
## 
## [[2]]
## # A tibble: 46 x 2
##    dest  num_flights
##    <chr>       <int>
##  1 ATL         17215
##  2 AUS          2439
##  3 BNA          6333
##  4 BOS         15508
##  5 BTV          2589
##  6 BUF          4681
##  7 CHS          2884
##  8 CLE          4573
##  9 CLT         14064
## 10 CMH          3524
## # i 36 more rows
\end{verbatim}

\hypertarget{question-5.}{%
\paragraph{Question 5.}\label{question-5.}}

Using the nycflights13 dataset, find the flight information (flight
number, origin, destination, carrier, number of flights in the year, and
percent late) for the flight numbers with the highest percentage of
arrival delays. Only include the flight numbers that have over 100
flights in the year.

\begin{Shaded}
\begin{Highlighting}[]
\NormalTok{flight\_info }\OtherTok{\textless{}{-}}\NormalTok{ flights }\SpecialCharTok{\%\textgreater{}\%}
  \FunctionTok{group\_by}\NormalTok{(flight) }\SpecialCharTok{\%\textgreater{}\%}
  \FunctionTok{filter}\NormalTok{(}\FunctionTok{n}\NormalTok{() }\SpecialCharTok{\textgreater{}} \DecValTok{100}\NormalTok{) }\SpecialCharTok{\%\textgreater{}\%}
  \FunctionTok{summarise}\NormalTok{(}
    \AttributeTok{origin =} \FunctionTok{first}\NormalTok{(origin),}
    \AttributeTok{dest =} \FunctionTok{first}\NormalTok{(dest),}
    \AttributeTok{carrier =} \FunctionTok{first}\NormalTok{(carrier),}
    \AttributeTok{num\_flights =} \FunctionTok{n}\NormalTok{(),}
    \AttributeTok{percent\_late =} \FunctionTok{mean}\NormalTok{(arr\_delay }\SpecialCharTok{\textgreater{}} \DecValTok{0}\NormalTok{) }\SpecialCharTok{*} \DecValTok{100}
\NormalTok{  ) }\SpecialCharTok{\%\textgreater{}\%}
  \FunctionTok{arrange}\NormalTok{(}\FunctionTok{desc}\NormalTok{(percent\_late))}

\NormalTok{flight\_info}
\end{Highlighting}
\end{Shaded}

\begin{verbatim}
## # A tibble: 1,157 x 6
##    flight origin dest  carrier num_flights percent_late
##     <int> <chr>  <chr> <chr>         <int>        <dbl>
##  1     43 JFK    MCO   B6              133         63.2
##  2    803 JFK    SJU   B6              142         61.3
##  3   1165 EWR    LAX   UA              147         61.2
##  4   1127 EWR    SFO   UA              107         58.9
##  5    195 EWR    MDW   WN              142         58.5
##  6    705 JFK    SJU   B6              225         53.8
##  7   1202 EWR    MIA   UA              106         53.8
##  8    137 JFK    RSW   B6              153         53.6
##  9    141 JFK    PBI   B6              377         52.8
## 10   1130 LGA    IAH   UA              103         49.5
## # i 1,147 more rows
\end{verbatim}

\end{document}
