% Options for packages loaded elsewhere
\PassOptionsToPackage{unicode}{hyperref}
\PassOptionsToPackage{hyphens}{url}
%
\documentclass[
]{article}
\usepackage{amsmath,amssymb}
\usepackage{iftex}
\ifPDFTeX
  \usepackage[T1]{fontenc}
  \usepackage[utf8]{inputenc}
  \usepackage{textcomp} % provide euro and other symbols
\else % if luatex or xetex
  \usepackage{unicode-math} % this also loads fontspec
  \defaultfontfeatures{Scale=MatchLowercase}
  \defaultfontfeatures[\rmfamily]{Ligatures=TeX,Scale=1}
\fi
\usepackage{lmodern}
\ifPDFTeX\else
  % xetex/luatex font selection
\fi
% Use upquote if available, for straight quotes in verbatim environments
\IfFileExists{upquote.sty}{\usepackage{upquote}}{}
\IfFileExists{microtype.sty}{% use microtype if available
  \usepackage[]{microtype}
  \UseMicrotypeSet[protrusion]{basicmath} % disable protrusion for tt fonts
}{}
\makeatletter
\@ifundefined{KOMAClassName}{% if non-KOMA class
  \IfFileExists{parskip.sty}{%
    \usepackage{parskip}
  }{% else
    \setlength{\parindent}{0pt}
    \setlength{\parskip}{6pt plus 2pt minus 1pt}}
}{% if KOMA class
  \KOMAoptions{parskip=half}}
\makeatother
\usepackage{xcolor}
\usepackage[margin=1in]{geometry}
\usepackage{color}
\usepackage{fancyvrb}
\newcommand{\VerbBar}{|}
\newcommand{\VERB}{\Verb[commandchars=\\\{\}]}
\DefineVerbatimEnvironment{Highlighting}{Verbatim}{commandchars=\\\{\}}
% Add ',fontsize=\small' for more characters per line
\usepackage{framed}
\definecolor{shadecolor}{RGB}{248,248,248}
\newenvironment{Shaded}{\begin{snugshade}}{\end{snugshade}}
\newcommand{\AlertTok}[1]{\textcolor[rgb]{0.94,0.16,0.16}{#1}}
\newcommand{\AnnotationTok}[1]{\textcolor[rgb]{0.56,0.35,0.01}{\textbf{\textit{#1}}}}
\newcommand{\AttributeTok}[1]{\textcolor[rgb]{0.13,0.29,0.53}{#1}}
\newcommand{\BaseNTok}[1]{\textcolor[rgb]{0.00,0.00,0.81}{#1}}
\newcommand{\BuiltInTok}[1]{#1}
\newcommand{\CharTok}[1]{\textcolor[rgb]{0.31,0.60,0.02}{#1}}
\newcommand{\CommentTok}[1]{\textcolor[rgb]{0.56,0.35,0.01}{\textit{#1}}}
\newcommand{\CommentVarTok}[1]{\textcolor[rgb]{0.56,0.35,0.01}{\textbf{\textit{#1}}}}
\newcommand{\ConstantTok}[1]{\textcolor[rgb]{0.56,0.35,0.01}{#1}}
\newcommand{\ControlFlowTok}[1]{\textcolor[rgb]{0.13,0.29,0.53}{\textbf{#1}}}
\newcommand{\DataTypeTok}[1]{\textcolor[rgb]{0.13,0.29,0.53}{#1}}
\newcommand{\DecValTok}[1]{\textcolor[rgb]{0.00,0.00,0.81}{#1}}
\newcommand{\DocumentationTok}[1]{\textcolor[rgb]{0.56,0.35,0.01}{\textbf{\textit{#1}}}}
\newcommand{\ErrorTok}[1]{\textcolor[rgb]{0.64,0.00,0.00}{\textbf{#1}}}
\newcommand{\ExtensionTok}[1]{#1}
\newcommand{\FloatTok}[1]{\textcolor[rgb]{0.00,0.00,0.81}{#1}}
\newcommand{\FunctionTok}[1]{\textcolor[rgb]{0.13,0.29,0.53}{\textbf{#1}}}
\newcommand{\ImportTok}[1]{#1}
\newcommand{\InformationTok}[1]{\textcolor[rgb]{0.56,0.35,0.01}{\textbf{\textit{#1}}}}
\newcommand{\KeywordTok}[1]{\textcolor[rgb]{0.13,0.29,0.53}{\textbf{#1}}}
\newcommand{\NormalTok}[1]{#1}
\newcommand{\OperatorTok}[1]{\textcolor[rgb]{0.81,0.36,0.00}{\textbf{#1}}}
\newcommand{\OtherTok}[1]{\textcolor[rgb]{0.56,0.35,0.01}{#1}}
\newcommand{\PreprocessorTok}[1]{\textcolor[rgb]{0.56,0.35,0.01}{\textit{#1}}}
\newcommand{\RegionMarkerTok}[1]{#1}
\newcommand{\SpecialCharTok}[1]{\textcolor[rgb]{0.81,0.36,0.00}{\textbf{#1}}}
\newcommand{\SpecialStringTok}[1]{\textcolor[rgb]{0.31,0.60,0.02}{#1}}
\newcommand{\StringTok}[1]{\textcolor[rgb]{0.31,0.60,0.02}{#1}}
\newcommand{\VariableTok}[1]{\textcolor[rgb]{0.00,0.00,0.00}{#1}}
\newcommand{\VerbatimStringTok}[1]{\textcolor[rgb]{0.31,0.60,0.02}{#1}}
\newcommand{\WarningTok}[1]{\textcolor[rgb]{0.56,0.35,0.01}{\textbf{\textit{#1}}}}
\usepackage{graphicx}
\makeatletter
\def\maxwidth{\ifdim\Gin@nat@width>\linewidth\linewidth\else\Gin@nat@width\fi}
\def\maxheight{\ifdim\Gin@nat@height>\textheight\textheight\else\Gin@nat@height\fi}
\makeatother
% Scale images if necessary, so that they will not overflow the page
% margins by default, and it is still possible to overwrite the defaults
% using explicit options in \includegraphics[width, height, ...]{}
\setkeys{Gin}{width=\maxwidth,height=\maxheight,keepaspectratio}
% Set default figure placement to htbp
\makeatletter
\def\fps@figure{htbp}
\makeatother
\setlength{\emergencystretch}{3em} % prevent overfull lines
\providecommand{\tightlist}{%
  \setlength{\itemsep}{0pt}\setlength{\parskip}{0pt}}
\setcounter{secnumdepth}{-\maxdimen} % remove section numbering
\ifLuaTeX
  \usepackage{selnolig}  % disable illegal ligatures
\fi
\IfFileExists{bookmark.sty}{\usepackage{bookmark}}{\usepackage{hyperref}}
\IfFileExists{xurl.sty}{\usepackage{xurl}}{} % add URL line breaks if available
\urlstyle{same}
\hypersetup{
  pdftitle={Functions Lab},
  hidelinks,
  pdfcreator={LaTeX via pandoc}}

\title{Functions Lab}
\author{}
\date{\vspace{-2.5em}}

\begin{document}
\maketitle

\hypertarget{assignment-instructions}{%
\subparagraph{Assignment Instructions}\label{assignment-instructions}}

Complete all questions below. After completing the assignment, knit your
document, and download both your .Rmd and knitted output. Upload your
files for peer review.

For each response, include comments detailing your response and what
each line does. Ensure you test your functions with sufficient test
cases to identify and correct any potential bugs.

\begin{center}\rule{0.5\linewidth}{0.5pt}\end{center}

\hypertarget{question-1.}{%
\subparagraph{Question 1.}\label{question-1.}}

Review the roll functions from Section 2 in \emph{Hands-On Programming
in R}. Using these functions as an example, create a function that
produces a histogram of 50,000 rolls of three 8 sided dice. Each die is
loaded so that the number 7 has a higher probability of being rolled
than the other numbers, assume all other sides of the die have a 1/10
probability of being rolled.

Your function should contain the arguments \texttt{max\_rolls},
\texttt{sides}, and \texttt{num\_of\_dice}. You may wish to set some of
the arguments to default values.

\begin{Shaded}
\begin{Highlighting}[]
\CommentTok{\# Function to simulate loaded dice rolls and produce histogram}
\NormalTok{simulate\_loaded\_dice\_rolls }\OtherTok{\textless{}{-}} \ControlFlowTok{function}\NormalTok{(}\AttributeTok{max\_rolls =} \DecValTok{50000}\NormalTok{, }\AttributeTok{sides =} \DecValTok{8}\NormalTok{, }\AttributeTok{num\_of\_dice =} \DecValTok{3}\NormalTok{) \{}
  
  \CommentTok{\# Define probabilities for each side of the die}
\NormalTok{  probabilities }\OtherTok{\textless{}{-}} \FunctionTok{rep}\NormalTok{(}\DecValTok{1}\SpecialCharTok{/}\DecValTok{10}\NormalTok{, sides)}
\NormalTok{  probabilities[}\DecValTok{7}\NormalTok{] }\OtherTok{\textless{}{-}} \DecValTok{2}\SpecialCharTok{/}\DecValTok{10}  \CommentTok{\# Setting higher probability for number 7}
  
  \CommentTok{\# Function to roll a single loaded die}
\NormalTok{  roll\_loaded\_die }\OtherTok{\textless{}{-}} \ControlFlowTok{function}\NormalTok{() \{}
    \FunctionTok{sample}\NormalTok{(}\DecValTok{1}\SpecialCharTok{:}\NormalTok{sides, }\AttributeTok{size =} \DecValTok{1}\NormalTok{, }\AttributeTok{prob =}\NormalTok{ probabilities)}
\NormalTok{  \}}
  
  \CommentTok{\# Function to roll multiple dice and sum their outcomes}
\NormalTok{  roll\_loaded\_dice }\OtherTok{\textless{}{-}} \ControlFlowTok{function}\NormalTok{(num\_dice) \{}
    \FunctionTok{sum}\NormalTok{(}\FunctionTok{replicate}\NormalTok{(num\_dice, }\FunctionTok{roll\_loaded\_die}\NormalTok{()))}
\NormalTok{  \}}
  
  \CommentTok{\# Simulate rolls}
\NormalTok{  rolls }\OtherTok{\textless{}{-}} \FunctionTok{replicate}\NormalTok{(max\_rolls, }\FunctionTok{roll\_loaded\_dice}\NormalTok{(num\_of\_dice))}
  
  \CommentTok{\# Plot histogram}
  \FunctionTok{hist}\NormalTok{(rolls, }\AttributeTok{breaks =} \FunctionTok{seq}\NormalTok{(}\FunctionTok{min}\NormalTok{(rolls), }\FunctionTok{max}\NormalTok{(rolls) }\SpecialCharTok{+} \DecValTok{1}\NormalTok{, }\AttributeTok{by =} \DecValTok{1}\NormalTok{), }
       \AttributeTok{main =} \FunctionTok{paste}\NormalTok{(}\StringTok{"Histogram of"}\NormalTok{, max\_rolls, }\StringTok{"rolls of"}\NormalTok{, num\_of\_dice, }\StringTok{"dice"}\NormalTok{),}
       \AttributeTok{xlab =} \StringTok{"Sum of Dice Rolls"}\NormalTok{,}
       \AttributeTok{ylab =} \StringTok{"Frequency"}\NormalTok{)}
\NormalTok{\}}

\CommentTok{\# Example usage:}
\FunctionTok{simulate\_loaded\_dice\_rolls}\NormalTok{()}
\end{Highlighting}
\end{Shaded}

\includegraphics{Functions-Lab_files/figure-latex/unnamed-chunk-1-1.pdf}

\hypertarget{question-2.}{%
\subparagraph{Question 2.}\label{question-2.}}

Write a function, rescale01(), that recieves a vector as an input and
checks that the inputs are all numeric. If the input vector is numeric,
map any -Inf and Inf values to 0 and 1, respectively. If the input
vector is non-numeric, stop the function and return the message ``inputs
must all be numeric''.

Be sure to thoroughly provide test cases. Additionally, ensure to allow
your response chunk to return error messages.

\begin{Shaded}
\begin{Highlighting}[]
\CommentTok{\# Function to rescale numeric vector to [0, 1]}
\NormalTok{rescale01 }\OtherTok{\textless{}{-}} \ControlFlowTok{function}\NormalTok{(x) \{}
  \CommentTok{\# Check if all elements of x are numeric}
  \ControlFlowTok{if}\NormalTok{ (}\SpecialCharTok{!}\FunctionTok{all}\NormalTok{(}\FunctionTok{is.numeric}\NormalTok{(x))) \{}
    \FunctionTok{stop}\NormalTok{(}\StringTok{"inputs must all be numeric"}\NormalTok{)}
\NormalTok{  \}}
  
  \CommentTok{\# Replace {-}Inf with 0 and Inf with 1}
\NormalTok{  x[x }\SpecialCharTok{==} \SpecialCharTok{{-}}\ConstantTok{Inf}\NormalTok{] }\OtherTok{\textless{}{-}} \DecValTok{0}
\NormalTok{  x[x }\SpecialCharTok{==} \ConstantTok{Inf}\NormalTok{] }\OtherTok{\textless{}{-}} \DecValTok{1}
  
  \FunctionTok{return}\NormalTok{(x)}
\NormalTok{\}}

\CommentTok{\# Test cases}
\CommentTok{\# Case 1: Numeric vector with {-}Inf and Inf}
\NormalTok{input1 }\OtherTok{\textless{}{-}} \FunctionTok{c}\NormalTok{(}\FloatTok{0.5}\NormalTok{, }\DecValTok{1}\NormalTok{, }\SpecialCharTok{{-}}\ConstantTok{Inf}\NormalTok{, }\DecValTok{2}\NormalTok{, }\ConstantTok{Inf}\NormalTok{, }\DecValTok{3}\NormalTok{)}
\NormalTok{result1 }\OtherTok{\textless{}{-}} \FunctionTok{rescale01}\NormalTok{(input1)}
\FunctionTok{print}\NormalTok{(result1)}
\end{Highlighting}
\end{Shaded}

\begin{verbatim}
## [1] 0.5 1.0 0.0 2.0 1.0 3.0
\end{verbatim}

\begin{Shaded}
\begin{Highlighting}[]
\CommentTok{\# Expected output: 0.5  1.0  0.0  2.0  1.0  3.0}

\CommentTok{\# Case 2: Numeric vector without {-}Inf and Inf}
\NormalTok{input2 }\OtherTok{\textless{}{-}} \FunctionTok{c}\NormalTok{(}\FloatTok{0.1}\NormalTok{, }\FloatTok{0.8}\NormalTok{, }\FloatTok{0.3}\NormalTok{, }\FloatTok{0.9}\NormalTok{)}
\NormalTok{result2 }\OtherTok{\textless{}{-}} \FunctionTok{rescale01}\NormalTok{(input2)}
\FunctionTok{print}\NormalTok{(result2)}
\end{Highlighting}
\end{Shaded}

\begin{verbatim}
## [1] 0.1 0.8 0.3 0.9
\end{verbatim}

\begin{Shaded}
\begin{Highlighting}[]
\CommentTok{\# Expected output: 0.1  0.8  0.3  0.9}

\CommentTok{\# Case 3: Numeric vector with only Inf}
\NormalTok{input3 }\OtherTok{\textless{}{-}} \FunctionTok{c}\NormalTok{(}\ConstantTok{Inf}\NormalTok{, }\ConstantTok{Inf}\NormalTok{, }\ConstantTok{Inf}\NormalTok{)}
\NormalTok{result3 }\OtherTok{\textless{}{-}} \FunctionTok{rescale01}\NormalTok{(input3)}
\FunctionTok{print}\NormalTok{(result3)}
\end{Highlighting}
\end{Shaded}

\begin{verbatim}
## [1] 1 1 1
\end{verbatim}

\begin{Shaded}
\begin{Highlighting}[]
\CommentTok{\# Expected output: 1.0  1.0  1.0}

\CommentTok{\# Case 4: Numeric vector with only {-}Inf}
\NormalTok{input4 }\OtherTok{\textless{}{-}} \FunctionTok{c}\NormalTok{(}\SpecialCharTok{{-}}\ConstantTok{Inf}\NormalTok{, }\SpecialCharTok{{-}}\ConstantTok{Inf}\NormalTok{, }\SpecialCharTok{{-}}\ConstantTok{Inf}\NormalTok{)}
\NormalTok{result4 }\OtherTok{\textless{}{-}} \FunctionTok{rescale01}\NormalTok{(input4)}
\FunctionTok{print}\NormalTok{(result4)}
\end{Highlighting}
\end{Shaded}

\begin{verbatim}
## [1] 0 0 0
\end{verbatim}

\begin{Shaded}
\begin{Highlighting}[]
\CommentTok{\# Expected output: 0.0  0.0  0.0}

\CommentTok{\# Case 5: Non{-}numeric vector}
\NormalTok{input5 }\OtherTok{\textless{}{-}} \FunctionTok{c}\NormalTok{(}\StringTok{"a"}\NormalTok{, }\StringTok{"b"}\NormalTok{, }\StringTok{"c"}\NormalTok{)}
\FunctionTok{tryCatch}\NormalTok{(\{}
\NormalTok{  result5 }\OtherTok{\textless{}{-}} \FunctionTok{rescale01}\NormalTok{(input5)}
  \FunctionTok{print}\NormalTok{(result5)}
\NormalTok{\}, }\AttributeTok{error =} \ControlFlowTok{function}\NormalTok{(e) \{}
  \FunctionTok{print}\NormalTok{(}\FunctionTok{paste}\NormalTok{(}\StringTok{"Error:"}\NormalTok{, e))}
\NormalTok{\})}
\end{Highlighting}
\end{Shaded}

\begin{verbatim}
## [1] "Error: Error in rescale01(input5): inputs must all be numeric\n"
\end{verbatim}

\begin{Shaded}
\begin{Highlighting}[]
\CommentTok{\# Expected output: Error: inputs must all be numeric}
\end{Highlighting}
\end{Shaded}

\hypertarget{question-3.}{%
\subparagraph{Question 3.}\label{question-3.}}

Write a function that takes two vectors of the same length and returns
the number of positions that have an NA in both vectors. If the vectors
are not the same length, stop the function and return the message
``vectors must be the same length''.

\begin{Shaded}
\begin{Highlighting}[]
\CommentTok{\# Function to count NA values in the same positions of two vectors}
\NormalTok{count\_na\_both }\OtherTok{\textless{}{-}} \ControlFlowTok{function}\NormalTok{(vec1, vec2) \{}
  \CommentTok{\# Check if vectors are of the same length}
  \ControlFlowTok{if}\NormalTok{ (}\FunctionTok{length}\NormalTok{(vec1) }\SpecialCharTok{!=} \FunctionTok{length}\NormalTok{(vec2)) \{}
    \FunctionTok{stop}\NormalTok{(}\StringTok{"vectors must be the same length"}\NormalTok{)}
\NormalTok{  \}}
  
  \CommentTok{\# Count positions where both vectors have NA}
\NormalTok{  num\_na\_both }\OtherTok{\textless{}{-}} \FunctionTok{sum}\NormalTok{(}\FunctionTok{is.na}\NormalTok{(vec1) }\SpecialCharTok{\&} \FunctionTok{is.na}\NormalTok{(vec2))}
  
  \FunctionTok{return}\NormalTok{(num\_na\_both)}
\NormalTok{\}}

\CommentTok{\# Test cases}
\CommentTok{\# Case 1: Vectors with NA at same positions}
\NormalTok{vec1 }\OtherTok{\textless{}{-}} \FunctionTok{c}\NormalTok{(}\DecValTok{1}\NormalTok{, }\ConstantTok{NA}\NormalTok{, }\DecValTok{3}\NormalTok{, }\ConstantTok{NA}\NormalTok{, }\DecValTok{5}\NormalTok{)}
\NormalTok{vec2 }\OtherTok{\textless{}{-}} \FunctionTok{c}\NormalTok{(}\ConstantTok{NA}\NormalTok{, }\DecValTok{2}\NormalTok{, }\ConstantTok{NA}\NormalTok{, }\DecValTok{4}\NormalTok{, }\ConstantTok{NA}\NormalTok{)}
\NormalTok{result1 }\OtherTok{\textless{}{-}} \FunctionTok{count\_na\_both}\NormalTok{(vec1, vec2)}
\FunctionTok{print}\NormalTok{(result1)}
\end{Highlighting}
\end{Shaded}

\begin{verbatim}
## [1] 0
\end{verbatim}

\begin{Shaded}
\begin{Highlighting}[]
\CommentTok{\# Expected output: 2 (positions 2 and 5 have NA in both vectors)}

\CommentTok{\# Case 2: Vectors with no NA at same positions}
\NormalTok{vec3 }\OtherTok{\textless{}{-}} \FunctionTok{c}\NormalTok{(}\DecValTok{1}\NormalTok{, }\DecValTok{2}\NormalTok{, }\DecValTok{3}\NormalTok{, }\DecValTok{4}\NormalTok{, }\DecValTok{5}\NormalTok{)}
\NormalTok{vec4 }\OtherTok{\textless{}{-}} \FunctionTok{c}\NormalTok{(}\ConstantTok{NA}\NormalTok{, }\ConstantTok{NA}\NormalTok{, }\ConstantTok{NA}\NormalTok{, }\ConstantTok{NA}\NormalTok{, }\ConstantTok{NA}\NormalTok{)}
\NormalTok{result2 }\OtherTok{\textless{}{-}} \FunctionTok{count\_na\_both}\NormalTok{(vec3, vec4)}
\FunctionTok{print}\NormalTok{(result2)}
\end{Highlighting}
\end{Shaded}

\begin{verbatim}
## [1] 0
\end{verbatim}

\begin{Shaded}
\begin{Highlighting}[]
\CommentTok{\# Expected output: 0 (no positions have NA in both vectors)}

\CommentTok{\# Case 3: Vectors of different lengths}
\NormalTok{vec5 }\OtherTok{\textless{}{-}} \FunctionTok{c}\NormalTok{(}\DecValTok{1}\NormalTok{, }\DecValTok{2}\NormalTok{, }\DecValTok{3}\NormalTok{)}
\NormalTok{vec6 }\OtherTok{\textless{}{-}} \FunctionTok{c}\NormalTok{(}\ConstantTok{NA}\NormalTok{, }\ConstantTok{NA}\NormalTok{, }\ConstantTok{NA}\NormalTok{, }\ConstantTok{NA}\NormalTok{)}
\FunctionTok{tryCatch}\NormalTok{(\{}
\NormalTok{  result3 }\OtherTok{\textless{}{-}} \FunctionTok{count\_na\_both}\NormalTok{(vec5, vec6)}
  \FunctionTok{print}\NormalTok{(result3)}
\NormalTok{\}, }\AttributeTok{error =} \ControlFlowTok{function}\NormalTok{(e) \{}
  \FunctionTok{print}\NormalTok{(}\FunctionTok{paste}\NormalTok{(}\StringTok{"Error:"}\NormalTok{, e))}
\NormalTok{\})}
\end{Highlighting}
\end{Shaded}

\begin{verbatim}
## [1] "Error: Error in count_na_both(vec5, vec6): vectors must be the same length\n"
\end{verbatim}

\begin{Shaded}
\begin{Highlighting}[]
\CommentTok{\# Expected output: Error: vectors must be the same length}
\end{Highlighting}
\end{Shaded}

\hypertarget{question-4}{%
\subparagraph{Question 4}\label{question-4}}

Implement a fizzbuzz function. It takes a single number as input. If the
number is divisible by three, it returns ``fizz''. If it's divisible by
five it returns ``buzz''. If it's divisible by three and five, it
returns ``fizzbuzz''. Otherwise, it returns the number.

\begin{Shaded}
\begin{Highlighting}[]
\NormalTok{fizzbuzz }\OtherTok{\textless{}{-}} \ControlFlowTok{function}\NormalTok{(number) \{}
  \ControlFlowTok{if}\NormalTok{ (number }\SpecialCharTok{\%\%} \DecValTok{3} \SpecialCharTok{==} \DecValTok{0} \SpecialCharTok{\&\&}\NormalTok{ number }\SpecialCharTok{\%\%} \DecValTok{5} \SpecialCharTok{==} \DecValTok{0}\NormalTok{) \{}
    \FunctionTok{return}\NormalTok{(}\StringTok{"fizzbuzz"}\NormalTok{)}
\NormalTok{  \} }\ControlFlowTok{else} \ControlFlowTok{if}\NormalTok{ (number }\SpecialCharTok{\%\%} \DecValTok{3} \SpecialCharTok{==} \DecValTok{0}\NormalTok{) \{}
    \FunctionTok{return}\NormalTok{(}\StringTok{"fizz"}\NormalTok{)}
\NormalTok{  \} }\ControlFlowTok{else} \ControlFlowTok{if}\NormalTok{ (number }\SpecialCharTok{\%\%} \DecValTok{5} \SpecialCharTok{==} \DecValTok{0}\NormalTok{) \{}
    \FunctionTok{return}\NormalTok{(}\StringTok{"buzz"}\NormalTok{)}
\NormalTok{  \} }\ControlFlowTok{else}\NormalTok{ \{}
    \FunctionTok{return}\NormalTok{(}\FunctionTok{as.character}\NormalTok{(number))}
\NormalTok{  \}}
\NormalTok{\}}

\CommentTok{\# Test cases}
\FunctionTok{print}\NormalTok{(}\FunctionTok{fizzbuzz}\NormalTok{(}\DecValTok{3}\NormalTok{))    }\CommentTok{\# Output: "fizz"}
\end{Highlighting}
\end{Shaded}

\begin{verbatim}
## [1] "fizz"
\end{verbatim}

\begin{Shaded}
\begin{Highlighting}[]
\FunctionTok{print}\NormalTok{(}\FunctionTok{fizzbuzz}\NormalTok{(}\DecValTok{5}\NormalTok{))    }\CommentTok{\# Output: "buzz"}
\end{Highlighting}
\end{Shaded}

\begin{verbatim}
## [1] "buzz"
\end{verbatim}

\begin{Shaded}
\begin{Highlighting}[]
\FunctionTok{print}\NormalTok{(}\FunctionTok{fizzbuzz}\NormalTok{(}\DecValTok{15}\NormalTok{))   }\CommentTok{\# Output: "fizzbuzz"}
\end{Highlighting}
\end{Shaded}

\begin{verbatim}
## [1] "fizzbuzz"
\end{verbatim}

\begin{Shaded}
\begin{Highlighting}[]
\FunctionTok{print}\NormalTok{(}\FunctionTok{fizzbuzz}\NormalTok{(}\DecValTok{7}\NormalTok{))    }\CommentTok{\# Output: "7"}
\end{Highlighting}
\end{Shaded}

\begin{verbatim}
## [1] "7"
\end{verbatim}

\hypertarget{question-5}{%
\subparagraph{Question 5}\label{question-5}}

Rewrite the function below using \texttt{cut()} to simplify the set of
nested if-else statements.

\begin{verbatim}
get_temp_desc <- function(temp) {
  if (temp <= 0) {
    "freezing"
  } else if (temp <= 10) {
    "cold"
  } else if (temp <= 20) {
    "cool"
  } else if (temp <= 30) {
    "warm"
  } else {
    "hot"
  } 
}
\end{verbatim}

\begin{Shaded}
\begin{Highlighting}[]
\NormalTok{get\_temp\_desc }\OtherTok{\textless{}{-}} \ControlFlowTok{function}\NormalTok{(temp) \{}
\NormalTok{  temperature\_labels }\OtherTok{\textless{}{-}} \FunctionTok{c}\NormalTok{(}\StringTok{"freezing"}\NormalTok{, }\StringTok{"cold"}\NormalTok{, }\StringTok{"cool"}\NormalTok{, }\StringTok{"warm"}\NormalTok{, }\StringTok{"hot"}\NormalTok{)}
\NormalTok{  cutpoints }\OtherTok{\textless{}{-}} \FunctionTok{c}\NormalTok{(}\SpecialCharTok{{-}}\ConstantTok{Inf}\NormalTok{, }\DecValTok{0}\NormalTok{, }\DecValTok{10}\NormalTok{, }\DecValTok{20}\NormalTok{, }\DecValTok{30}\NormalTok{, }\ConstantTok{Inf}\NormalTok{)}
  
\NormalTok{  desc }\OtherTok{\textless{}{-}} \FunctionTok{cut}\NormalTok{(temp, }\AttributeTok{breaks =}\NormalTok{ cutpoints, }\AttributeTok{labels =}\NormalTok{ temperature\_labels)}
  
  \FunctionTok{return}\NormalTok{(}\FunctionTok{as.character}\NormalTok{(desc))}
\NormalTok{\}}

\CommentTok{\# Test cases}
\FunctionTok{print}\NormalTok{(}\FunctionTok{get\_temp\_desc}\NormalTok{(}\SpecialCharTok{{-}}\DecValTok{5}\NormalTok{))   }\CommentTok{\# Output: "freezing"}
\end{Highlighting}
\end{Shaded}

\begin{verbatim}
## [1] "freezing"
\end{verbatim}

\begin{Shaded}
\begin{Highlighting}[]
\FunctionTok{print}\NormalTok{(}\FunctionTok{get\_temp\_desc}\NormalTok{(}\DecValTok{8}\NormalTok{))    }\CommentTok{\# Output: "cold"}
\end{Highlighting}
\end{Shaded}

\begin{verbatim}
## [1] "cold"
\end{verbatim}

\begin{Shaded}
\begin{Highlighting}[]
\FunctionTok{print}\NormalTok{(}\FunctionTok{get\_temp\_desc}\NormalTok{(}\DecValTok{15}\NormalTok{))   }\CommentTok{\# Output: "cool"}
\end{Highlighting}
\end{Shaded}

\begin{verbatim}
## [1] "cool"
\end{verbatim}

\begin{Shaded}
\begin{Highlighting}[]
\FunctionTok{print}\NormalTok{(}\FunctionTok{get\_temp\_desc}\NormalTok{(}\DecValTok{25}\NormalTok{))   }\CommentTok{\# Output: "warm"}
\end{Highlighting}
\end{Shaded}

\begin{verbatim}
## [1] "warm"
\end{verbatim}

\begin{Shaded}
\begin{Highlighting}[]
\FunctionTok{print}\NormalTok{(}\FunctionTok{get\_temp\_desc}\NormalTok{(}\DecValTok{35}\NormalTok{))   }\CommentTok{\# Output: "hot"}
\end{Highlighting}
\end{Shaded}

\begin{verbatim}
## [1] "hot"
\end{verbatim}

\end{document}
